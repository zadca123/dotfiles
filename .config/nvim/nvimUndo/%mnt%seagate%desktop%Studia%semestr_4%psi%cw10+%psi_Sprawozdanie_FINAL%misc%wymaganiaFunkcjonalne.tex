Vim�UnDo��b>P��RN����v>;L�sV�p�!��7ᡧ
`�_�_�����`�_���5�_�����V`�_����\begin{itemize}�  \item \textbf{Zarządzaj koszykiem} --- Klient ma możliwość przeglądania  zawartości koszyka, zmieniania ilości zawartych w nim produktów oraz usuwania poszczególnych pozycji z koszyka (CRUD).�  \item \textbf{Wybierz produkt} --- usługa polega na prezentowaniu podzielonej na grupy i podgrupy listy produktów elektronicznych, które można zakupić w sklepie internetowym. Klient wybiera opcję Wyszukaj produkt, w celu zawężenia listy wyświetlanych produktów. Po kliknięciu fotografi Klient ma możliwość zapoznania się z jego opisem (opis, fotografia, specyfikacja, cena, opinie) i może go dodać do koszyka. (wyświetl, wyszukaj, wybierz, pokaż szczegóły, pokaz opinię).�  \item \textbf{Zarządzaj koszykiem} --- Klient ma możliwość przeglądania  zawartości koszyka, zmieniania ilości zawartych w nim produktów oraz usuwania poszczególnych pozycji z koszyka (CRUD).c  \item \textbf{Złoż zamówienia} --- złożenie zamówienia przeznaczone jest dla autoryzowanych klientów. W ramach składania zamówienie Klient zobowiązany jest wybrać formę płatności, dane płatnika i sposób dostarczenia. Z perspektywy sklepu można wyróżnić następujące mechanizmy płatności: płatność online (klient może wykon po złożeniu zamówienia przechodząc do zadania realizowania płatności online), płatność za pobraniem (operator systemu wprowadza dane o pobraniu --- numer zamówienia, kwota, data do systemu). (akceptuj, anuluj, obsłuż i weryfikuj wprowadzane dane).u  \item \textbf{Drukuj fakturę sprzedaży} --- funkcja drukująca fakturę sprzedaży (generuj pdf, drukuj, anuluj).�  \item \textbf{Realizuj płatność online} --- płatność online można zrealizować na trzy sposoby – obsługa płatności karta kredytowa, obsługa płatność w systemie payU, PayPal, ePrzelew (obsłuż wybór sposobu płatności).�  \item \textbf{Zaksięguj płatność za pobraniem} --- operator systemu wprowadza dane o pobraniu- numer zamówienia, kwota, data do systemu (rejestruj dane, anuluj).;  \item \textbf{Autoryzuj użytkownika} --- obsługa logowania do systemu na podstawie danych użytkowania (email, hasło) oraz rejestracja konta nowego użytkownika i obsługa trybu przypomnienia  hasła. (obsłuż logowanie, rejestracja użytkownika, obsługa przypomnienia hasła, obsługa włamania do systemu).|  \item \textbf{Zarządzaj produktem} --- przygotowywanie dostępnych produktu do sprzedaży przed operatora systemu (CRUD).�  \item \textbf{Zarządzaj użytkownikami} --- operator systemu ma administracyjne możliwości zaradzania danymi użytkowników (CRUD).�  \item \textbf{Zarządzaj zamówieniami} --- funkcja dla operatora systemu do przeglądania złożonych zamówień i awaryjnych zmian statusu zamówienia (przeglądaj, anuluj).
\end{itemize}5��